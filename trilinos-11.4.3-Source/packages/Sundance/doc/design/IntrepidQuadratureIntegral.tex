\documentclass{article}

\author{Robert C. Kirby, Kevin R. Long and Bart van Bloemen Waanders}
\title{Redesign of the \texttt{QuadratureIntegral} class to support non-scalar functions}

\begin{document}
\maketitle

\section{Introduction}
As the symbolic system within Sundance is extended to support 
\( H(\mathrm{div}) \) and \( H(\mathrm{curl}) \) elements, it is
necessary for the element integration routines to support a wider
range of integrands.  This document attempts to enumerate the types
of computations necessary.

To establish some terminology, we divide expressions functionals,
linear forms, and bilinear forms.  A functional has
no test or trial functions and simply returns a number.  A linear form
constructs a vector, and it contains a test function but no unknown function.
This paradigm is needed for 
\begin{enumerate}
\item Residual evaluation for Newton's method
\item Matrix-free application of a stiffness matrix
\item Formation of the load vector
\end{enumerate}
Finally, a bilinear form provides a matrix over each mesh entity and must
contain a test and trial function.

Nonlinearities, etc, are handled by the symbolic and discrete function
systems and are already tabulated at quadrature points before being
passed into the quadrature integral evaluation.

\section{Functionals}
Nothing to do.

\section{Linear forms}
For linear forms, we may distinguish several cases, based on which space
the test function \( v \) lives in (\(H^1\), \( H(\mathrm{div}) \), or
\( H(\mathrm{curl}) \).
\subsection{Linear forms with \( H^1 \) test functions}
For the current version of Sundance (not recognizing gradients as
such), we will need to support operations of the form:
\[
\int f v
\]
and
\[
\int f D_i v,
\]
where the integral could be over a maximal cell or perhaps a
particular facet.  \( D_i \) is the partial derivative operator in
a particular Cartesian direction.

In the future, we will have vector-valued
functions, and will need to support linear forms of the form
\[
\int \tilde{f} \cdot \nabla v,
\]
where \( \tilde{f} \) is some known vector.  We will also want to
support
\[
\int f (\nabla u \cdot n)
\]
over boundary facets.  This is properly a scalar operation, but relies
on recognizing the gradient as a particular entity.


\subsection{Linear forms with \( H(\mathrm{div}) \) test functions}
For now, Sundance will only recognize scalar entities, so we need 
\[
\int f v_i,
\]
where \( v_i \) is the \(i^\mathrm{th}\) component of a vector in \(
H(\mathrm{div}) \).  We will also need
\[
\int f ( \nabla \cdot \tilde{v} )
\]
on cells and
\[
\int f (\tilde{v} \cdot n)
\]
on 1-cofacets, where \( n \) is a normal vector.  These last
expressions are nice because they correspond to boundary degrees of
freedom in the \( H(\mathrm{div}) \) finite element spaces.

With full vector-valued symbolics, we also want to support
\[
\int f \cdot \tilde{v}
\]

Note that general traces are not well-defined on \( H(\mathrm{div}) \),
so we need to be careful in what we allow.  However, DG-type methods
and \emph{a posteriori} error estimates also sometimes require
tangential components of these fields on boundaries, so supporting
these operations as well makes sense.

\subsection{Linear forms on \(H(\mathrm{curl})\)}


\section{Bilinear forms}
There is the potential for some combinatorial blow-up in the cases we
need to handle for bilinear forms, as we must support pairs of test and trial
spaces.  

\subsection{\( H^1 \) test and trial spaces}
Currently, if both test and trial spaces are \( H^1 \), we
will need to support 
\[
\int f u v
\]
\[
\int f u D_j v
\]
\[
\int f D_i u v
\]
\[
\int f D_i u D_j v
\]

Eventually, this will need to be extended to support
\[
\int f \nabla u \cdot \nabla v
\]
and
\[
\int ( A \nabla u ) \cdot \nabla v,
\]
where \( A \) is some tensor.

\subsection{\( H^1 \) trial and \( H(\mathrm{div}) \) test space}
For now, for trial function \( u \in H^1 \) and test function \(
\tilde{v} \in H(\mathrm{div}) \), we need to support operations
\[
\int f u v_i
\]
where \( v_i \) is a component of an \( H(\mathrm{div}) \) function and
\[
\int f u \left( \nabla \cdot \tilde{v} \right).
\]
We also need
\[
\int f u (\tilde{v} \cdot n)
\]
on facets.

In the future, we need more combinatorial explosion to handle
gradients times vectors, etc.  We also need to put \( H(\mathrm{curl})
\) in the mix.

\end{document}